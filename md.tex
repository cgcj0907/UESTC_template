% 综设报告:report|课程论文:course|学士论文:bachelor|硕士论文:master|博士论文:doctor
\documentclass[master]{thesis-uestc}

% =====================================================
% 修复 Pandoc 列表 \tightlist 未定义错误
% =====================================================
\providecommand{\tightlist}{%
  \setlength{\itemsep}{0pt}%
  \setlength{\parskip}{0pt}%
}

% =====================================================
% Pandoc 代码块兼容支持(Shaded / Highlighting)
% =====================================================
\usepackage{color}
\usepackage{fancyvrb}
\usepackage{framed}
\usepackage{xcolor}

% 定义代码块背景颜色
\definecolor{shadecolor}{RGB}{248,248,248}

% 定义 Shaded 环境(代码块背景)
\makeatletter
\@ifundefined{Shaded}{%
  \newenvironment{Shaded}{\begin{snugshade}}{\end{snugshade}}
}{}
\makeatother

% 定义 Highlighting 环境(Pandoc 内部使用)
\DefineVerbatimEnvironment{Highlighting}{Verbatim}{
  commandchars=\\\{\},
  fontsize=\small,
  baselinestretch=1,
}

% 定义 Pandoc 高亮命令
\newcommand{\AlertTok}[1]{\textcolor[rgb]{0.94,0.16,0.16}{#1}}
\newcommand{\AnnotationTok}[1]{\textcolor[rgb]{0.38,0.63,0.69}{#1}}
\newcommand{\AttributeTok}[1]{\textcolor[rgb]{0.77,0.63,0.00}{#1}}
\newcommand{\BaseNTok}[1]{\textcolor[rgb]{0.00,0.00,0.81}{#1}}
\newcommand{\BuiltInTok}[1]{\textcolor[rgb]{0.50,0.00,0.50}{#1}}
\newcommand{\CharTok}[1]{\textcolor[rgb]{0.31,0.60,0.02}{#1}}
\newcommand{\CommentTok}[1]{\textcolor[rgb]{0.56,0.35,0.01}{#1}}
\newcommand{\ConstantTok}[1]{\textcolor[rgb]{0.00,0.00,0.81}{#1}}
\newcommand{\ControlFlowTok}[1]{\textcolor[rgb]{0.00,0.25,0.75}{#1}}
\newcommand{\DataTypeTok}[1]{\textcolor[rgb]{0.13,0.29,0.53}{#1}}
\newcommand{\DecValTok}[1]{\textcolor[rgb]{0.00,0.00,0.81}{#1}}
\newcommand{\ExtensionTok}[1]{\textcolor[rgb]{0.00,0.25,0.75}{#1}}
\newcommand{\FloatTok}[1]{\textcolor[rgb]{0.00,0.00,0.81}{#1}}
\newcommand{\FunctionTok}[1]{\textcolor[rgb]{0.00,0.25,0.75}{#1}}
\newcommand{\ImportTok}[1]{\textcolor[rgb]{0.00,0.50,0.50}{#1}}
\newcommand{\KeywordTok}[1]{\textcolor[rgb]{0.00,0.25,0.75}{\textbf{#1}}}
\newcommand{\NormalTok}[1]{#1}
\newcommand{\OperatorTok}[1]{\textcolor[rgb]{0.00,0.25,0.75}{#1}}
\newcommand{\OtherTok}[1]{\textcolor[rgb]{0.56,0.35,0.01}{#1}}
\newcommand{\PreprocessorTok}[1]{\textcolor[rgb]{0.38,0.63,0.69}{#1}}
\newcommand{\StringTok}[1]{\textcolor[rgb]{0.31,0.60,0.02}{#1}}
\newcommand{\VariableTok}[1]{\textcolor[rgb]{0.00,0.25,0.75}{#1}}
\newcommand{\WarningTok}[1]{\textcolor[rgb]{0.94,0.16,0.16}{#1}}

% =====================================================
% 基本信息
% =====================================================
\title{时域积分方程时间步进算法及其快速算法}{The Time Marching Scheme of Time Domain Integral Equation and Corresponding Fast Algorithm}

\author{王稳}{Wang Wen}                       
\advisor{赖生建\chinesespace 副教授}{Dr. Shengjian Lai}   
\school{物理电子学院}{School of Physical Electronics}   
\major{无线电物理}{Radio Physics}          
\studentnumber{201421040223}              

% =====================================================
\begin{document}

% ===============================
% 封面
% ===============================
\makecover

% ===============================
% 中文摘要
% ===============================
\begin{chineseabstract}
……摘要(中文)

\chinesekeyword{...关键词(中文)}
\end{chineseabstract}

% ===============================
% 英文摘要
% ===============================
\begin{englishabstract}
.....摘要(英文)

\englishkeyword{...关键词(英文)}
\end{englishabstract}

% ===============================
% 目录
% ===============================
\thesistableofcontents

% ===============================
% 正文章节
% $body$ 用于 Pandoc 插入正文内容
% ===============================
$body$

% ===============================
% 致谢
% ===============================
\thesisacknowledgement
...致谢词

% ===============================
% 附录
% ===============================
\thesisappendix

\chapter{附录标题}           % 附录章标题
\section{附录节标题}       % 附录节标题

% ===============================
% 参考文献
% ===============================
% 列出所有文献条目(即使未引用),取消下面注释
\nocite{*}

\thesisbibliography{reference}  % 模板自带命令加载参考文献
% 使用原生 bibtex 命令示例:
% \bibliographystyle{thesis-uestc}
% \bibliography{reference}

% ===============================
% 论文成果
% ===============================
\thesisaccomplish{publications}
...成果

% ===============================
% 外文资料原文
% ===============================
\thesistranslationoriginal
外文资料原文

% ===============================
% 外文资料译文
% ===============================
\thesistranslationchinese
外文资料译文

\end{document}
